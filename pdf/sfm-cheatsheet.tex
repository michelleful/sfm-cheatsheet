\documentclass[10pt,landscape]{article}
\usepackage{multicol}
\usepackage{calc}
\usepackage{ifthen}
\usepackage[landscape]{geometry}
\usepackage{amsmath,amsthm,amsfonts,amssymb}
\usepackage{color,graphicx,overpic}
\usepackage{hyperref}
\usepackage{enumitem}
\usepackage{amsmath}

\pdfinfo{
  /Title (Street-Fighting Mathematics)
  /Author (Michelle Fullwood)
  /Subject (Street-Fighting Mathematics)
  /Keywords (mathematics, shortcuts, maths, math, cheatsheet, street-fighting mathematics, edX, sanjoy mahajan)}

% This sets page margins to .5 inch if using letter paper, and to 1cm
% if using A4 paper. (This probably isn't strictly necessary.)
% If using another size paper, use default 1cm margins.
\ifthenelse{\lengthtest { \paperwidth = 11in}}
    { \geometry{top=.5in,left=.5in,right=.5in,bottom=.5in} }
    {\ifthenelse{ \lengthtest{ \paperwidth = 297mm}}
        {\geometry{top=1cm,left=1cm,right=1cm,bottom=1cm} }
        {\geometry{top=1cm,left=1cm,right=1cm,bottom=1cm} }
    }

% Turn off header and footer
\pagestyle{empty}

% Redefine itemize settings to use less space
\setlist[itemize]{noitemsep, topsep=0pt, leftmargin=*}
\def\labelitemi{\checkmark}

% Redefine section commands to use less space
\makeatletter
\renewcommand{\section}{\@startsection{section}{1}{0mm}%
                                {-1ex plus -.5ex minus -.2ex}%
                                {0.5ex plus .2ex}%x
                                {\normalfont\large\bfseries}}
\renewcommand{\subsection}{\@startsection{subsection}{2}{0mm}%
                                {-1explus -.5ex minus -.2ex}%
                                {0.5ex plus .2ex}%
                                {\normalfont\normalsize\bfseries}}
\renewcommand{\subsubsection}{\@startsection{subsubsection}{3}{0mm}%
                                {-1ex plus -.5ex minus -.2ex}%
                                {1ex plus .2ex}%
                                {\normalfont\small\bfseries}}
\makeatother

% Define BibTeX command
\def\BibTeX{{\rm B\kern-.05em{\sc i\kern-.025em b}\kern-.08em
    T\kern-.1667em\lower.7ex\hbox{E}\kern-.125emX}}

% Don't print section numbers
\setcounter{secnumdepth}{0}


\setlength{\parindent}{0pt}
\setlength{\parskip}{0pt plus 0.5ex}

%My Environments
\newtheorem{example}[section]{Example}
% -----------------------------------------------------------------------

\begin{document}
\raggedright
\footnotesize
\begin{multicols}{3}


% multicol parameters
% These lengths are set only within the two main columns
%\setlength{\columnseprule}{0.25pt}
\setlength{\premulticols}{1pt}
\setlength{\postmulticols}{1pt}
\setlength{\multicolsep}{1pt}
\setlength{\columnsep}{2pt}

\begin{center}
     \Large{\underline{Street-Fighting Mathematics}} \\
\end{center}

\begin{itemize}
 \item Most material is from Sanjoy Mahajan's edX course 6.SFMx and book {\it Street-Fighting Mathematics}
\end{itemize}

\section{Estimating Quantities}

\begin{itemize}
 \item Break it down into parts you can estimate
 \item Don't be afraid to lump -- errors will cancel out
 \item Make sure the dimensions work out
\end{itemize}

\subsection{Multiplying large numbers [5.1]}
\begin{itemize}
 \item Convert numbers to scientific notation $k \cdot 10^{n}$ 
 \item Lump $k$ into 1 (0 to 1.8), {\it few} (1.8 to 5.6), 10 (5.6 to 10)
 \item {\it few}$^2$ = 10
 \item Add powers of multiplicands $n_1, n_2, ...$
\end{itemize}

\subsection{Fractional changes [5.2]}

 $$(m + \Delta m \%)(n + \Delta n \%) \approx m \cdot n + (\Delta m + \Delta n)\%$$

 $$\frac{\Delta x^n}{x^n} \approx n \cdot \frac{\Delta x}{x}$$ 

\begin{itemize}
 \item \begin{tabular}{ccc}
Ex: $m'$ & Big part ($m$) & $\Delta m$ \\
5.3   &     5 +      & $0.3/5 \approx 6\%$ \\
3.04  &     3 +      & $0.04/3 \approx 1.3\%$ \\ \hline
      &    15 +      & $7.3\% \cdot 15 \approx 1.1 = 16.1$ \\
\end{tabular}
 \item Ex: 4\% increase in sides of rectangle $\rightarrow$ 8\% increase in area
 \item Ex: inflation by 10\% + discount of 15\% $\rightarrow$ 5\% decrease
 \item Ex: division, $n=-1$. $1/13 \approx 1/10 - 30\% = 0.07$
 \item Ex: square roots, $n=1/2$. $\sqrt{10} \approx \sqrt{9} + (1/2)(1/9) \approx 3.17$ 
\end{itemize}

\subsection{Improving fractional estimations}

\begin{itemize}
 \item Increase the accuracy of the big part, e.g. by multiplying both numerator and denominator
       by a convenient number.
 \item Ex: $1/13 \cdot 8/8 = 8/104 \approx 8/100 - 4\% = 0.0768$
 \item In the case of division, results in a quadratic improvement: 
       $\frac{\Delta x^{-1}}{x^{-1}} \approx -1 \cdot \frac{\Delta x}{x} + (-1)^2 \cdot \frac{\Delta x}{x}^2 $
 \item Ex: $\sqrt{10} = \sqrt{360}/6 \approx \sqrt{361}/6 = 19/6 \approx 3.167$
 \item Polya's method: $\sqrt{2} = \sqrt{4/3}/\sqrt{2/3} \approx (1+(1/6))/(1-(1/6)) = 7/5 \approx 1.4$
 \item Another ex: $\ln 2 = \ln\frac{4}{3} - \ln\frac{2}{3} \approx \frac{1}{3} - (-\frac{1}{3}) = \frac{2}{3}$
\end{itemize}

\subsection{Useful approximations}

\begin{itemize}
 \item  $(1+z)^n \approx 1+nz$, $z \ll 1$ and $nz \ll 1$
 \item  $(1+z)^n \approx e^{nz}$, $z \ll 1$ and $nz$ unrestricted
 \item  $\ln(1+z) \approx z$, small $z$
 \item  $\ln(2) \approx 0.7$
 \item  $\ln(10) \approx 2.3$
 \item Ex: 5\% bacteria mutated per round, how many unmutated after 140 rounds? 
       $0.95^{140} = (1-1/20)^{140} \approx e^{-140/20} \approx e^{-6.9} \approx 10^{-3} = 0.001$
 \item $\sin \theta \approx \theta$ at small angles
 \item $\cos \theta \approx 1-\frac{\theta^2}{2}$ at small angles
 \item Rule of 72: if a quantity increases $x\%$ per time unit (e.g. year), it doubles in $72/x$ time units (years).
\end{itemize}

\subsection{Estimating from bounds}
\begin{itemize}
 \item Estimate a lower bound $l$ and an upper bound $u$ for the desired quantity
 \item Your estimate = GeometricMean$(l,u)$ = $\sqrt{l \cdot u}$
\end{itemize}

\section{Guessing a Formula}

\subsection{Dimensional analysis [1]}

\begin{itemize}
 \item You can only make meaningful comparisons with items of the same dimension.
 \item Gather together the elements you think play a role in the formula.
 \item Compare dimensions to guess at formula.
 \item Ex: Impact speed $v$: dim $LT^{-1}$, $g$: dim $LT^{-2}$, height $h$: dim $L$.
           $v \sim \sqrt{g \cdot h}$
 \item Special cases:
   \begin{itemize}
     \item Exponents must be dimensionless.
     \item $d/dt$ = `a little bit of $t$': dim $T$
     \item $d^{2}t$ = `a little bit of a little bit of $t$': dim $T$
     \item $dt^2$ = dim $T^{2}$ 
     \item Summation and integration signs do not affect dimensions.
   \end{itemize}
 \item Warning: there will sometimes be $>1$ solution! 
 \item Take advantage of symmetries. Consider solutions such as $a^2+ab+b^2$.
 \item Additive terms must all have the same dimensions.
 \item To judge between solutions find the constant term, try easy cases.
\end{itemize}

\subsection{Easy cases [2]}
\begin{itemize}
 \item Try $x = 0, 1, y, \infty$
 \item Reduce dimensions
 \item When $n$ has arbitrary range, try a small number of terms
\end{itemize}

\section{Solving formulae}

\subsection{Quadratic equations [5.HW]}

$$ ax^2 + bx + c = 0$$

\begin{itemize}
 \item $x \approx 0$. Remove small $ax^2$ term, $x \approx -c/b$
 \item $x \gg 1$. Remove small $c$, $x \approx -b/a$
 \item Substitute existing estimates of $x$ into eqn to iterate better solutions
\end{itemize}

\subsection{Evaluating integrals}
\begin{itemize}
 \item Can you use one of the approximations above to simplify the integrand? (ideally to exponential)
 \item Don't be afraid to extend the bounds to infinity
 \item $\int^{\infty}_{-\infty}e^{-\alpha t^2} \mathrm{d}t = \sqrt{\pi/\alpha}$ 
 \item Archimedes' theorem: an inverted parabola encloses 2/3 of a rectangle's area 
 \item $1/e$ technique: Find the width $\Delta$x that produces a change of $e$, use the resulting rectangle area as an estimate.
 
 \begin{center}
 \includegraphics[scale=1]{e_lumping.png} 
 \end{center}
 \vspace{-1em}
 \item Full-Width Half-Maximum (FWHM) technique: find the width where the $y$-value is half the maximum.

 \begin{center}
 \includegraphics[scale=1]{fwhm_lumping.png} 
 \end{center}

 \end{itemize}
 
\subsection{Evaluating derivatives}
\begin{itemize}
 \item Significant change approximation:
 
 $$\frac{df}{dx} \sim \frac{\textnormal{significant $\Delta$f near x}}{\textnormal{$\Delta$x}}$$
 
 \item $\frac{d^2x}{dt^2} \sim \frac{\textnormal{significant $\Delta x$}}{(\Delta t \textnormal{that produces a significant} \Delta x)^2}$ 
 
\end{itemize}

\subsection{Find the convergence point of a recurrence relation}
\begin{itemize}
 \item Substitute $x$ for $x_n$ and $x_{n-1}$ etc, and solve.
 \item Ex: $x_{n+1} = \frac{1}{2}(x_n + \frac{2}{x_n})$, $n\geq 0$
 \item Solve with: $x = \frac{1}{2}(x + \frac{2}{x}) \rightarrow x = \sqrt{2}$
\end{itemize}

\subsection{Finding a closed formula for a summation}
\begin{itemize}
 \item Use the Euler-MacLaurin summation formula:

 $$ \begin{aligned} 
 \sum^{b}_{a}f(k) = \int^{b}_{a} f(k) dk + \frac{f(b)+f(a)}{2} + \frac{f'(b)-f'(a)}{12} - \\ \frac{f^{(3)}(b)-f^{(3)}(a)}{720} + ...
  \end{aligned} $$
\end{itemize}

% You can even have references
\rule{0.3\linewidth}{0.25pt}
\scriptsize
%\bibliographystyle{abstract}
%\bibliography{refFile}

Layout by dror on StackOverflow
\end{multicols}
\end{document}